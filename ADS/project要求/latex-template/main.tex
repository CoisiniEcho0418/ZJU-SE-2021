%
% This is the LaTeX template file for lecture notes for ADS Zhejiang University.  
% When preparing LaTeX notes for this class, please use this template.
%
% To familiarize yourself with this template, the body contains
% some examples of its use.  Look them over.  Then you can
% run LaTeX on this file.  After you have LaTeXed this file then
% you can look over the result either by printing it out with
% dvips or using xdvi.
%
% This template is based on the template for Prof. Sinclair's CS 270.

\documentclass[twoside]{article}
\usepackage{CJKutf8}

%\usepackage{graphics}
\usepackage{lineno}
\linenumbers
\usepackage{graphics}
\usepackage{geometry}
\usepackage{forest,amsmath}
\usepackage{enumerate}
\geometry{left=2.5cm,right=2cm,top=2.5cm,bottom=2.5cm}

%\setlength{\oddsidemargin}{0.25 in}
%\setlength{\evensidemargin}{-0.25 in}
%\setlength{\topmargin}{-0.6 in}
%\setlength{\textwidth}{6.5 in}
%\setlength{\textheight}{8.5 in}
%\setlength{\headsep}{0.75 in}
\setlength{\parindent}{0 in}
\setlength{\parskip}{0.1 in}

\usepackage{listings}
\usepackage{color}
\renewcommand\lstlistingname{Quelltext} % Change language of section name
\lstset{ % General setup for the package
    language= C,
    %basicstyle=\small\sffamily,
    basicstyle=\ttfamily,
    numbers=left,
     numberstyle=\tiny,
    frame=tb,
    tabsize=4,
    columns=fixed,
    showstringspaces=false,
    showtabs=false,
    keepspaces,
    commentstyle=\color{red},
    keywordstyle=\color{blue}
}

%
% The following commands set up the lecnum (lecture number)
% counter and make various numbering schemes work relative
% to the lecture number.
%
%\newcounter{lecnum}
%\renewcommand{\thepage}{\thelecnum-\arabic{page}}
%\renewcommand{\thesection}{\thelecnum.\arabic{section}}
%\renewcommand{\theequation}{\thelecnum.\arabic{equation}}
%\renewcommand{\thefigure}{\thelecnum.\arabic{figure}}
%\renewcommand{\thetable}{\thelecnum.\arabic{table}}

%
% The following macro is used to generate the header.
%


%


%Use this command for a figure; it puts a figure in wherever you want it.
%usage: 
%\begin{figure}
%\begin{center}
%\includegraphics[width=5in]{fig-file}
%\caption{}\label{fig:delavl}
%\end{center}
%\end{figure}

%%% Use the following command for a table
%%%

% Use these for theorems, lemmas, proofs, etc.
\newtheorem{theorem}{Theorem}[theorem]
\newtheorem{lemma}[theorem]{Lemma}
\newtheorem{proposition}[theorem]{Proposition}
\newtheorem{claim}[theorem]{Claim}
\newtheorem{corollary}[theorem]{Corollary}
\newtheorem{definition}[theorem]{Definition}
\newenvironment{proof}{{\bf Proof:}}{\hfill\rule{2mm}{2mm}}

% **** IF YOU WANT TO DEFINE ADDITIONAL MACROS FOR YOURSELF, PUT THEM HERE:

\begin{document}
\begin{CJK*}{UTF8}{gbsn}
%FILL IN THE RIGHT INFO.
%\lecture{**LECTURE-NUMBER**}{**DATE**}{**LECTURER**}{**SCRIBE**}
%\lecture{1}{Project Name}{Deshi Ye}{Student 1, Student 2, 学生3}
%\footnotetext{These notes are partially based on those of Nigel Mansell.}
\title{Title of Project Report}
\author{Group number: Student 1 \and Student 2 \and 学生3}
\maketitle
% **** YOUR NOTES GO HERE:

% Some general latex examples and examples making use of the
% macros follow.  
%**** IN GENERAL, BE BRIEF. LONG SCRIBE NOTES, NO MATTER HOW WELL WRITTEN,
%**** ARE NEVER READ BY ANYBODY.
\begin{abstract}
An abstract is a short summary of a research paper or a report.
Abstract is to help your reader understand quickly what your report is about.
Abstract typically spends:
\begin{itemize}
    \item importance of research and your problem (Introduction)
    \item what you did (Methods)
    \item what you have found (Results)
    \item discuss the implication and significance of the project (Discussion)
\end{itemize}
\end{abstract}

\section{Introduction}

Problem description, purpose of this report, and (if any) background of the data structures and the algorithms.  Be concise.

\section{Data Structure / Algorithm Specification}

Description (pseudo-code preferred) of all the algorithms involved for solving the problem,  
including specifications of main data structures.
Or, if you are to introduce a new data structure and its related operations, do it in this chapter.

If the algorithm you used is adopted from previous work, please cite that work, such as our textbook~\cite{10.5555/1614191}. 

\section{Testing Results}

Table of test cases.  
Each test case usually consists of a brief description of the purpose of this case, the expected result, 
the actual behavior of your program, the possible cause of a bug if your program does not function as expected, 
and the current status (“pass”, or “corrected”, or “pending”).

Example table in Table~\ref{tab:relevance}:
\begin{table}[h!]
  \begin{center}
    \caption{Relevance}
    \label{tab:relevance}
    \begin{tabular}{|l|c|r|} % <-- Alignments: 1st column left, 2nd middle and 3rd right, with vertical lines in between
    \hline
       & \textbf{Relevant} & \textbf{Irrelevant}\\
      \hline
      Retrieved & $R_R$ (true positive) & $I_R$ (false positive) \\ \hline
      Not Retrieved & $R_N $ (false negative)& $I_N$ (true negative)\\
      \hline
    \end{tabular}
  \end{center}
\end{table}

\section{Analysis and Comments}

Analysis of the time and space complexities of the algorithms.  
Comments on comparing with other known data structures and algorithms.  
Further possible improvements.

\section{Author list}
Specify who did what to show that particular contributors deserve to have their names printed in the cover page of your report.

\section*{Declaration}
We hereby declare that all the work done in this project titled "XXX" is of our independent effort as a group.

\section{Signatures}

Each author must sign his/her name here.

\vskip 1.5in


Please keep in mind that these are the “minimum” requirements.  Other requirements will be specified according to each project assignment.
\newpage

%%%% References 
%%% bibliographystyle: is the referece style, such as plain, abbrv et al. 
%%% bibliography: the file.bib;
\bibliographystyle{abbrv}
%\bibliographystyle{plain}
\bibliography{ads}

\appendix 
\section{Source Code (if required)}
At least 30\% of the lines must be commented.  Otherwise the code will NOT be evaluated.

\begin{lstlisting}
        static Position
        SingleRotateWithLeft( Position K2 )
        {
            Position K1;

            K1 = K2->Left;
            K2->Left = K1->Right;
            K1->Right = K2;

            K2->Height = Max( Height( K2->Left ), Height( K2->Right ) ) + 1;
            K1->Height = Max( Height( K1->Left ), K2->Height ) + 1;

            return K1;  /* New root */
        }
\end{lstlisting}

Pseudocode of Dijkstra's algorithm (Heap-Based)
\begin{lstlisting}[mathescape=true]
for every vertex $v$ in $V$ do 
    $d_v \leftarrow \infty$; $p_v \leftarrow null$
    Insert($Q, v, d_v$)  /* initialize vertex priority in the priority queue */
$d_s \leftarrow 0$; Decrease($Q, s, d_s$)  /* update priority of $s$ with $d_s$ */
$V_T \leftarrow \emptyset$
for $i \leftarrow 0$ to $|V| - 1$ do
    $u^* \leftarrow DeleteMin(Q)$  /* delete the minimum priority element */
    $V_T \leftarrow V_t  \bigcup  \{u^*\}$
    for every vertex $u$ in $V - V_T$  that is adjacent to $u^*$ do
        if $d_{u^*} + w(u^*, u) < d_u $
            $d_u \leftarrow d_{u^*} + w(u^*, u)$;  $p_u \leftarrow u^*$
            Decrease($Q, u, d_u$)
\end{lstlisting}

\end{CJK*} 
\end{document}





